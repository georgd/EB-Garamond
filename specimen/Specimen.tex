\documentclass[pagesize,11.5pt,DIV14]{scrreprt}
\usepackage{fontspec}
\usepackage{microtype}
\usepackage[babelshorthands]{polyglossia}
\setdefaultlanguage{english}
\setotherlanguages{%
french,czech,polish,latin,italian,catalan,german,romanian,%
portuges,swedish,finnish,dutch,russian,serbian,ukrainian,%
bulgarian,slovenian,vietnamese%
}
\usepackage{csquotes}
\defaultfontfeatures{RawFeature={+calt,+clig,
+tlig}}%,Numbers={Proportional,OldStyle}}
\setmainfont%
{EB Garamond}
\setsansfont{Linux Biolinum O}
\usepackage{cmap}
\usepackage{multicol}
\usepackage{sectsty}
\allsectionsfont{\rmfamily}
\subsectionfont{\sc}
\paragraphfont{\normalsize\sc}
\begin{document}
{\pagestyle{empty}
\begin{center}
\pagestyle{empty}
\vspace*{5cm}
\fontsize{48}{48}{\addfontfeature{Color=980000}
EB Garamond}\\
\vspace*{2cm}
{\fontsize{24}{24}\addfontfeature{Color=000000} Claude Garamont’s designs go open source
\vfill\vfill
\today\\\vspace*{2cm}}
\fontsize{48}{48}{\addfontfeature{Color=980000}
\char"E001 \char"E002
}
\end{center}\clearpage}
\begin{multicols}{2}
\paragraph*{Claude Garamond} (ca. 1480 – 1561) was a Parisian publisher. He was one of the leading type designers  of his time, and is credited with the introduction of the apostrophe, the accent and the cedilla to the French language. Several contemporary typefaces, including those currently known as Garamond, Granjon, and Sabon, reflect his influence. Garamond was an apprentice of Simon de Colines; later, he was an assistant to Geoffroy Tory [sic!], whose interests in humanist typography and the ancient Greek capital letterforms, or majuscules, may have informed Garamond’s later work.

Garamond came to prominence in 1541, when three of his Greek typefaces (e.g. the Grecs du roi (1541)) were requested for a royally-ordered book series by Robert Estienne. Garamond based these types on the handwriting of Angelo Vergecio, the King’s Librarian at Fontainebleau, as well as that of his ten-year-old pupil, Henri Estienne. According to Arthur Tilley, the resulting books are “among the most finished specimens of typography that exist.” Shortly thereafter, Garamond created the Roman types for which […]\\
{\scriptsize From Wikipedia, the free encyclopedia}
\begin{catalan}
\paragraph*{Claude Garamond}\addfontfeature{Language=Catalan} (París, 1490 - 1561) va ser un tipògraf, impressor i gravador de matrius francès. La seva obra tipogràfica es considera clàssica dins de l’estil antic i d'inspiració per a composicions modernes.
El 1510 va començar el seu aprenentatge amb el tipògraf i impressor Antoine Augereau. Durant la primera meitat del segle XVI els impressors compartien totes les instàncies en l’elaboració d’un llibre des del disseny tipogràfic fins a l’enquadernació. Claude Garamond va ser el primer que es va especialitzar en el disseny, gravat i foneria de tipus com a servei a altres impressors ajudat pel seu aprenent Jacques Sabon. Les referències tipogràfiques de Garamond inclouen els treballs de Conrad Sweynheym, Arnold Pannartz, Erhard Ratdolt, Nicholas Jenson, Aldo Manuzio, Francesco Griffo, Henri, Robert, Ludovicio degli Arrighi de Venècia, Giovani Antonio Tagliente i Giovanbattista Palatino, una llista eclèctica d’experts coneguts per la seva excel·lència tipogràfica. A finals del 1520 Garamond va ser comissionat per a subministrar els seus tipus al famós impressor escolar Robert Estienne.\\
{\scriptsize De Viquipèdia}
\end{catalan}
\begin{german}
\addfontfeature{Language=German}
\paragraph*{Claude Garamond,} auch Garamont, (*1499 (oder 1490) in Paris; †\,1561 in Paris) war ein französischer Schriftgießer, Typograf, Stempelschneider und Verleger. Er schuf die noch heute verwendete Schriftart Garamond.

Claude Garamond lernte das Handwerk des Schriftschneidens bei Geoffroy Troy, er war Schüler und Mitarbeiter des Pariser Stempelschneiders und Druckers Antoine Augereau, einige der Arbeiten Garamonds werden in verschiedenen Quellen Augereau zugeordnet – in jedem Fall ist eine Einflussnahme seines Lehrers bis zu dessen Tod auf dem Scheiterhaufen im Jahr 1534 sehr wahrscheinlich.

Garamonds erste Antiqua-Schriften dürf"|ten um 1530/1531 entstanden sein, als erstmals in vier verschiedenen Pariser Druckereien Schriften eines neuen Typus auf"|tauchten, die Claude Garamond zugeschrieben werden konnten. Die Vorarbeiten hierzu sind bis ins Jahr 1525 zurückzudatieren.

Nimmt man als Referenz die Arbeiten für seinen wichtigsten Kunden, den Drucker Robert Estienne, […]\\
{\scriptsize Aus Wikipedia, der freien Enzyklopädie}
\end{german}

\begin{french}
\paragraph*{Claude Garamont,} souvent orthographié Garamond (à cause de son pseudonyme Garamondus), né en 1499 à Paris où il est mort en 1561, est un tailleur et fondeur de caractères et un imprimeur français.

Avec Guillaume~I Le Bé et Robert Granjon, il est un des plus fameux créateurs de caractères français du {\addfontfeature{RawFeature=+c2sc}XVI{\addfontfeature{RawFeature=+ordn}e}} siècle. Il est notamment le créateur des « Grecs du Roi », une série de polices grecques imitée de modèles manuscrits, ainsi que d’un fameux type romain qui porte son nom et qui sera abondamment copié tout au long de l’histoire.

Il a appris son métier en étant l’élève d’Antoine Augereau, un tailleur de caractères parisien qui s’était reconverti au métier de libraire et d’imprimeur.

En 1540, Pierre Duchâtel, conseiller et aumônier de François I{\addfontfeature{RawFeature=+ordn}er} commanda à Garamond les poinçons de trois sortes de caractères d’un alphabet grec aux frais de Robert Estienne (qui en fit usage pour ses éditions grecques, à partir de 1543). Pour dessiner ces caractères, dit plus tard Grecs du Roi, […]\\
{\scriptsize De Wikipédia, l'encyclopédie libre.}
\end{french}

\begin{portuges}
\paragraph*{Claude Garamond} (1480-1561) foi um editor francês.

As fontes criadas por Claude Garamond em Paris, a partir de 1530, refinadas dos tipos usados por Aldus Manutius, de 1455, são ainda hoje um referencial tipográfico forte, influenciando diversas interpretações em famílias de letras contemporâneas.

Como ficou provado no início deste século (1925) por Beatrice Warde (sob a pseudônimo Paul Beaujon), seriam todas essas “descendentes”, em verdade, baseadas nos caracteres da "l’université in the imprimerie royale" de paris.

A versão mais próxima da tipografia em estilo Garamond é a Granjon, de George William Jones, feita entre 1928 e 1931, possuindo este nome para evitar confusões com aquelas que se pretendem “originais”.\\
{\scriptsize Origem: Wikipédia, a enciclopédia livre.}
\end{portuges}
\begin{italian}
\paragraph*{Claude Garamond} (Parigi, 1499 – Parigi, 1561) è stato un tipografo francese, conosciuto anche con il nome italianizzato di Claudio Garamontio.

Si tratta forse del più famoso incisore di caratteri mobili francese. Creò il grec du Roi (greco del re), cioè il carattere greco con cui erano pubblicate le edizioni dei classici dedicati al re di Francia. Creò inoltre il "romano" una serie di caratteri latini che porta il suo nome (Garamond) e che sarà abbondantemente copiato durante la storia.\\
{\scriptsize Da Wikipedia, l'enciclopedia libera.}
\end{italian}
\begin{dutch}
\paragraph*{Claude Garamond} (Parijs, ca. 1480 - aldaar, 1561) was een Parijse graveur van druklettermodellen, uitgever en één van de toonaangevende letterontwerpers van zijn tijd. Sommige van zijn lettertypen zijn nog in gebruik, in het bijzonder het naar hemzelf genoemde Garamond. Ca. 1510 werd Claude Garamond leerlinggraveur bij de drukker Antoine Augereau en rond 1520 werkte hij bij Geoffroy Tory de professor, vertaler, schrijver, boekverkoper, drukker, ontwerper en illustrator. Vanaf ongeveer 1530 werkte Garamond voor zichzelf en vervaardigde hij kant en klare drukletters voor drukkers.\\
{\scriptsize Uit Wikipedia, de vrije encyclopedie.}
\end{dutch}
\begin{swedish}
\paragraph*{Claude Garamond} född ca 1490, död 1561, var en fransk typsnittsskapare.

Claude Garamond var troligen lärling hos stämpelskäraren Antoine Augereau. Sent 1520-tal blev Claude kontaktad av parisaren och tryckaren Robert Estienne som önskade en uppsättning antikvor. Dessa typer sågs för första gången i Paraphrasis in Elegantiarum Libros Laurentii Vallae. Många[vem?] anser att ursprunget till Claude Garamonds former kommer ifrån typer skurna åt Aldus Manutius av Francesco da Bologna med släktnamnet Griffo. Återstoden av Claude Garamonds matriser och patris finns numera i Plantin-Moretus Museum i Antwerpen, samt vid Imprimerie Nationale i Paris. Claude Garamond dog 1561.\\
{\scriptsize Från Wikipedia}
\end{swedish}
\begin{finnish}
\paragraph*{Claude Garamond} (n.1480–1561) oli ranskalainen kustantaja ja aikansa huomattavimpia kirjainmuotoilijoita.

Yhä edelleen on käytössä useita antiikva-kirjasintyyppejä, jotka perustuvat hänen suunnittelemilleen kirjaimille. Näihin kuuluvat esimerkiksi Garamond No. 3, Garamond No. 5, Adobe Garamond, Simoncini Garamond, ITC Garamond ja Granjon.
{\scriptsize Wikipedia}

\end{finnish}
\begin{latin} \addfontfeature{Language=Latin}
\paragraph*{\addfontfeature{Language=Latin}Claudius Garamond} (etiam Garamont natus Lutetiae anno 1499 (aut anno 1490) - ibidem mortuus est anno 1561) fuit typographus et editor Francicus, qui clarus est praecipue quia scripturam quae eius nomen Garamond fert creavit.\\
{\scriptsize E Vicipaedia}
\end{latin}
%\clearpage
\begin{czech}
\paragraph*{Sazba (od „sázeti“)} je původně tisková forma pro tisk z výšky (knihtisk), tvořená jednotlivými literami, případně obrázky. V širším slova smyslu hotová předloha pro tisk, zejména textová. Řemeslník, který vytvářel „horkou“ sazbu, se nazýval sazeč; s přechodem na modernější tiskové techniky, zejména ofset, pracuje grafik se sázecím programem na počítači.\par
Sestavování tiskové formy z pohyblivých kovových liter se objevilo už ve 12.-13. století v Číně, nejstarší dochovaný exemplář je korejská kniha Jikji z roku 1377. Ruční zhotovování liter však bylo nesmírně pracné a technika se užívala jen sporadicky. Kolem roku 1440 udělal mohučský zlatník Johannes Gutenberg několik převratných vynálezů, mezi něž patřilo i odlévání liter. Ručně vyrobený ocelový model litery se zakalil a vyrazil do ocelové matrice, do níž se pak odléval libovolný počet shodných liter. Už Gutenberg také objevil liteřinu, slitinu olova a antimonu, která spojuje nízký bod tání s větší pevností.\\
{\scriptsize Z Wikipedie, otevřené encyklopedie.}
\end{czech}
\begin{polish}
\paragraph*{Skład} – termin zecerski (dziś już historyczny) oznaczający tekst, który powstał fizycznie, czyli został ułożony z czcionek lub wierszy linotypowych (a także innych elementów, jak monotypy, linie czy justunek). Skład ma postać szpalty i będzie dopiero łamany. Składem jest także tabela, już złożona, ale jeszcze nie włamana w kolumnę.\par
Czynność, w wyniku której powstawał tekst uformowany w taki sposób, nazywano składaniem, i to niezależnie od tego, czy odbywała się ona całkowicie ręcznie, czy też pomagały w niej w pierwszej fazie maszyny odlewające linotypy i monotypy.\par
Podstawowymi cechami złożonego tekstu były: ustalony krój pisma w określonej odmianie i stopniu, oraz szerokość wierszy.\\
{\scriptsize Z Wikipedii, wolnej encyklopedii.}
\end{polish}
\begin{slovenian}
\paragraph*{Johannes Gensfleisch zur Laden zum Gutenberg,} nemški izumitelj, * okoli 1398, Mainz, Nemčija, † 3. februar 1468.\par
Gutenberg je postal slaven zaradi svojih prispevkov k tehnologiji tiskanja na sredini 15. stoletja. Velja za izumitelja tiska s premičnimi kovinskimi črkami. Izpopolnil je črnilo, zlitine za črke, šablono za natančnejše vstavljanje črk, naredil pa je tudi novo vrsto tiskarske stiskalnice, zasnovane na stiskalnici za grozdje.
{\scriptsize Iz Wikipedije, proste enciklopedije}
\end{slovenian}

\begin{vietnamese}
\paragraph*{Johannes Gutenberg} (khoảng năm 1390 – 3 tháng 2 năm 1468), là một công nhân đồng thời là một nhà phát minh người Đức. Ông trở nên nổi tiếng vì phát minh ra phương pháp in dấu vào năm những năm 1450.

Gutenberg sinh ở Mainz, nước Đức. Ông là con trai của một thương gia tên là Friele Gensfleisch zur Laden. Người cha của Gutenberg đã lấy "zum Gutenberg" là nơi họ đã sống lúc đó để đặt tên cho ông.

Gutenberg đã phát minh ra một loại hợp kim dùng cho việc in ấn; mực; và cách cố định chữ in (chữ kim loại) rất chính xác; và một loại máy in mới. Nhiều người cho rằng Gutenberg đã phát minh ra loại bản in mẫu trượt ở châu Âu, nhưng thực ra nó đã được phát minh ra ở Triều Tiên trước đó.
{\scriptsize Bách khoa toàn thư mở Wikipedia}
\end{vietnamese}
\begin{serbian}
\addfontfeature{RawFeature=+ss01}
\paragraph*{\addfontfeature{RawFeature=+ss01}\addfontfeature{Language=Serbian}Јохан Гутенберг} (нем. Johannes Gensfleisch zur Laden zum Gutenberg) (око 1400, Мајнц – 3.~фебруар 1468), је био немачки металски радник и сматра се проналазачем технике штампања металним помичним словима. Идеја те технике је била већ дуже позната али до тада није била усавршена у тој мери. Познати су и наводи да је та техника већ раније била реализована у Кини, Кореји и Јапану. Најстарија још учувана штампана књига датира из 868. године. Године 1041. се у Кини појављују и први примерци штампаних књига техником металних покретних слова и као проналазач се наводи извесни Би Шенг. Гутенбергова заслуга, као оца модерног штампарства, се састоји у усавршавању појединачних покретних слова, легура олова, антимона и калаја у техници високе штампе, али и тадашњи проналазак штампарске пресе.\par
О његовом животу је веома мало познато и велики дио информација су само претпоставке. Његов брат Фриле Генсфлајш (нем. Friele Gensfleisch) је рођен у Елтвилу где је вероватно, са Гутенбергом основао малу штампарију и где је провео цели свој животни век (1434.-1447.).\\
{\scriptsize Из Википедије, слободне енциклопедије}
\end{serbian}
\begin{russian}
\paragraph*{Клод Гарамо́н} (ок. 1500—1561) — парижский пуансонист, печатник, одна из важнейших фигур французского ренессанса. Он был учеником печатников Антуана Ожеро и Симона де Колина. Позже основал небольшую книгопечатню неподалеку от Сорбонны.\par
Впервые известность к Гарамону пришла в 1540-х гг., когда он вырезал grecs du roi — три греческих курсива для издания классиков, поддержанного королём. Позже Гарамон вырезал и другие шрифты, в том числе прямой. Его шрифты, основаны в первую очередь на альдовых, но достаточно оригинальны. В курсивах Гарамона впервые появляются наклонные прописные, а также т. н. росчерки (swash).\par
Спустя 60 лет после смерти Гарамона, пуансонист Жан Жаннон повторил прямой шрифт Гарамона, однако в формах более близких к барокко, чем к ренессансу. Именно этот шрифт, утерянный и забытый, был вновь найден в первой половине XIX в., и ошибочно приписан Клоду Гарамону. Ошибка обнаружилась в 1927 г., однако за 5 лет до этого фирма Monotype уже выпустила новую версию шрифта Жаннона под названием Garamond Roman. Именно это стало причиной того, что в XX в. под одним названием было выпущено несколько шрифтов, восходящих не только к двум разным авторам, но и к двум разным эпохам.
\\
{\scriptsize из Википедии — свободной энциклопедии}
\end{russian}
\begin{bulgarian}
\paragraph*{Йоханес Генсфлайш цур Ладен цум Гутенберг} (на немски: Johannes Gensfleisch zur Laden zum Gutenberg) е германски златар и печатар, основоположник на съвременното книгопечатане. Изобретеният от него машинен печат с подвижен набор поставя началото на революция в печатането и често е определяно като едно от най-важните събития на Новото време. Изобретението му оказва силно влияние върху развитието на Ренесанса, Реформацията и Научната революция и поставя материалните основи на съвременната икономика на знанието и демократизация на образованието.\\
{\scriptsize от Уикипедия, свободната енциклопедия}
\end{bulgarian}

\begin{ukrainian}
\paragraph*{Йо́ганн Гутенбе́рг} (Johann Gutenberg) (* 1397-1400,
Майнц (Німеччина) — † 3 лютого 1468) — європей-
ський винахідник і першодрукар, який винайшов і за-
провадив промислову технологію друкарства у пра-
ктику виготовлення книжок.
Прецизійний механік шляхетного роду.
1434-1444 — працював у Страсбурзі. Відомо, що в
цей час мав справу з судом через якийсь свій винахід,
який намагався тримати у таємниці.
Деякою мірою складання перегукується з верс-
ткою.
{\scriptsize з Вікіпедії — вільної енциклопедії.}
\end{ukrainian}

\paragraph*{Иоханнэс Гутэнбэрг} (Герман хэл: Johannes Gensfleisch
zur Laden zum Gutenberg; 1398 онд төрж 1468
оны 2 дугаар сарын 3-нд нас барсан) нь 1439 оны
үед Европт анх удаа зөөдөг үсэгтэй хэвлэлийн аргыг
хэрэглэсэн, дэлхийд анх удаа механик хэвлэлийн
суурь машин зохион бүтээсэн Германы алтны
дархан, хэвлэгч юм. Түүний томоохон бүтээл болох
Гутэнбэргийн библь нь (42 мөрт Библь гэдэг) гоо зүй,
техникийн чанараараа үнэлэгдсээр ирсэн.\\
{\scriptsize Чөлөөт нэвтэрхий толь, Википедиагаас}

\paragraph*{Ио́ганн Генсфляйш цур Ладен цум Гу́тенберг} (нем.
Johannes Gensfleisch zur Laden zum Gutenberg, 1397
и дон 1400 и лошты, Майнц — 3 февраль 1468,
Майнц) — немӹц кӹртни пӓшӓ мастар, ювелир,
книгӓм пецӓтлӹмӹ станым шанен лыкшы.\\
{\scriptsize Ирӹкӓн энциклопеди Википеди гӹц материал}


\paragraph*{Йоганн Ґутенберґ} (1397/1400, Могуч – 3. února
1468, Могуч) быв вынаходником технолоґії
механічного книгтиску помочов рушаючіх ся писмен.\\
{\scriptsize Матеріал з Вікіпедії}
\end{multicols}
%\end{landscape}
\clearpage
\fontsize{16}{16}
\begin{center}\addfontfeature{LetterSpace=10.0}
¿ABCDEFGHIJKLMNOPQQRSẞTUVWXYZÞÐ?\\
¡abcdefghijklmnopqrsſßtuvwxyzþðđf!\\
\$0123456789€ \XeTeXglyph 99 {\addfontfeature{Numbers=Lining}9876543210}¢\\
»«;:,.-–—›‹„“”‚‘’[…]\{\}()·×÷\\
"'\#\%\& *+/<=>@\textbackslash \textasciicircum \textasciitilde §©ªº°µΩ\\
\textsc{abcdefghijklmnopqrstuvwxyzß}\\ 
fffiffiflfbfhfjfkft\\
ffbffhffjffkfflfft\\
ſſſiſſiſlſbſhſjſkſt\\
ſſbſſhſſjſſkſſlſſt\\
\char"261E \addfontfeature{RawFeature=+dlig}ÆæstctœŒ\char"261C\\
{\addfontfeature{LetterSpace=0}
Text{\addfontfeature{RawFeature=+sups}abcdefghijklmnopqrstuvwxyz0123456789}x\\
Text{\addfontfeature{RawFeature=+ordn}abcdefghijklmnopqrstuvwxyz0123456789}x\\
Text{\addfontfeature{RawFeature=+subs}abcdefghijklmnopqrstuvwxyz0123456789}x\\
Text{\addfontfeature{RawFeature=+sinf}abcdefghijklmnopqrstuvwxyz0123456789}x}\\
ÁÀÃĄÄ{\addfontfeature{Language=German}Ä}%Ǎ
ÂĂÅÉÈ%Ẽ
ĘËĚÊĔÍÌĨĮÏ%Ǐ
ÎĬİ\\
Ó{\addfontfeature{Language=Polish}Ó}ÒÕǪÖ{\addfontfeature{Language=German}Ö}ǑÔŎŐØÚÙŨŲÜ{\addfontfeature{Language=German}Ü}ǓÛŬŮŰ\\
áàãąä%ǎ
âăåéè%ẽ
ęëěêĕíìĩįï%ǐ
îĭıóòõǫöǒôǒőøúùũųü%ǔ
ûŭůű\\
Ć{\addfontfeature{Language=Polish}Ć}ćÇçĎďĢģĜĝǦǧĞğĤĥḨḩĴĵĸŁłĽľÑñŇňŃ{\addfontfeature{Language=Polish}Ń}ńŊŋ\\
ŘřŔŕŚ{\addfontfeature{Language=Polish}Ś}śŠšŞşŤťŢţŴŵÝýŶŷŸÿŹ{\addfontfeature{Language=Polish}Ź}źŻżŽž\\
\addfontfeature{RawFeature={+smcp,+c2sc}}ÁÀÃĄÄǍÂĂÅÉÈẼĘËĚÊĔÍÌĨĮÏǏÎĬİ\\
ÓÒÕÖÔŎŐØÚÙŨŲÜÛŬŮŰ\\
áàãąäǎâăåéèẽęëěêĕíìĩįïǐîĭıóòõǫöǒôǒőøúùũųüǔûŭůű\\
ĆćÇçĎďĢģĜĝĞğĤĥĴĵĸŁłĽľÑñŇňŃńŊŋ\\
ŘřŔশŠšŞşŤťŢţŴŵÝýŶŷŸÿŹźŻżŽž\\
\Huge \char"E001 \char"B6 \char"E002
\vfill
\fontsize{12}{12}\scriptsize\textit{Georg Duffner | www.georgduffner.at/ebgaramond/ | g.duffner@gmail.com}
\end{center}
\end{document}